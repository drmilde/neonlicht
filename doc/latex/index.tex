\hypertarget{index_intro_sec}{}\section{Introduction}\label{index_intro_sec}
\hyperlink{classNeonlicht}{Neonlicht} is a synthesizer engine targeting the fast and easy creation of Sound\-Units (aka synthesizers and audio effects) that run on most Unix/\-Linux machines.

More specifically \hyperlink{classNeonlicht}{Neonlicht} is targeting the Raspberry 3 computer. As such it is optimized to be used with the low computing power of this machine.

I am aiming at using the engine as part of teaching students in the field of digital media. As such, I am trying to design a system, that is generally understandable, does not require to much of a prior knowledge in audio programming, but still offers enough flexibilty and power to create interesting and expressive musical instruments and audio effects. Using the Raspberry 3 is part of this approach. I want the software to be running on cheap hardware allowing students to experiment without the need of having access to expensive fruity systems.

\begin{DoxyAuthor}{Author}
J\-T\-M,  email\-: \href{mailto:milde@hs-fulda.de}{\tt milde@hs-\/fulda.\-de} 
\end{DoxyAuthor}
\begin{DoxyDate}{Date}
April 2016
\end{DoxyDate}
\hypertarget{index_install_sec}{}\section{Installation}\label{index_install_sec}
Installing \hyperlink{classNeonlicht}{Neonlicht} on your computer is done by taking the usual steps. The code is self contained, meaning all nessecary libraries are added to the source distribution of \hyperlink{classNeonlicht}{Neonlicht}.

In order to compile \hyperlink{classNeonlicht}{Neonlicht} you first compile the underlying libraries and, in a second step, compile \hyperlink{classNeonlicht}{Neonlicht} and the additional tools.

P\-L\-E\-A\-S\-E N\-O\-T\-E\-: Part of this documentation is in German ... Sorry for the inconvenience. I am working on it \-:)\hypertarget{index_step1}{}\subsection{Step 1\-: Compiling the libraries}\label{index_step1}
to be done ... 